% Options for packages loaded elsewhere
\PassOptionsToPackage{unicode}{hyperref}
\PassOptionsToPackage{hyphens}{url}
%
\documentclass[
  ignorenonframetext,
  aspectratio=32,
]{beamer}
\usepackage{pgfpages}
\setbeamertemplate{caption}[numbered]
\setbeamertemplate{caption label separator}{: }
\setbeamercolor{caption name}{fg=normal text.fg}
\beamertemplatenavigationsymbolshorizontal
% Prevent slide breaks in the middle of a paragraph
\widowpenalties 1 10000
\raggedbottom
\setbeamertemplate{part page}{
  \centering
  \begin{beamercolorbox}[sep=16pt,center]{part title}
    \usebeamerfont{part title}\insertpart\par
  \end{beamercolorbox}
}
\setbeamertemplate{section page}{
  \centering
  \begin{beamercolorbox}[sep=12pt,center]{section title}
    \usebeamerfont{section title}\insertsection\par
  \end{beamercolorbox}
}
\setbeamertemplate{subsection page}{
  \centering
  \begin{beamercolorbox}[sep=8pt,center]{subsection title}
    \usebeamerfont{subsection title}\insertsubsection\par
  \end{beamercolorbox}
}
\AtBeginPart{
  \frame{\partpage}
}
\AtBeginSection{
  \ifbibliography
  \else
    \frame{\sectionpage}
  \fi
}
\AtBeginSubsection{
  \frame{\subsectionpage}
}

\usepackage{amsmath,amssymb}
\usepackage{iftex}
\ifPDFTeX
  \usepackage[T1]{fontenc}
  \usepackage[utf8]{inputenc}
  \usepackage{textcomp} % provide euro and other symbols
\else % if luatex or xetex
  \usepackage{unicode-math}
  \defaultfontfeatures{Scale=MatchLowercase}
  \defaultfontfeatures[\rmfamily]{Ligatures=TeX,Scale=1}
\fi
\usepackage{lmodern}
\usetheme[]{default}
\ifPDFTeX\else  
    % xetex/luatex font selection
\fi
% Use upquote if available, for straight quotes in verbatim environments
\IfFileExists{upquote.sty}{\usepackage{upquote}}{}
\IfFileExists{microtype.sty}{% use microtype if available
  \usepackage[]{microtype}
  \UseMicrotypeSet[protrusion]{basicmath} % disable protrusion for tt fonts
}{}
\makeatletter
\@ifundefined{KOMAClassName}{% if non-KOMA class
  \IfFileExists{parskip.sty}{%
    \usepackage{parskip}
  }{% else
    \setlength{\parindent}{0pt}
    \setlength{\parskip}{6pt plus 2pt minus 1pt}}
}{% if KOMA class
  \KOMAoptions{parskip=half}}
\makeatother
\usepackage{xcolor}
\newif\ifbibliography
\setlength{\emergencystretch}{3em} % prevent overfull lines
\setcounter{secnumdepth}{-\maxdimen} % remove section numbering


\providecommand{\tightlist}{%
  \setlength{\itemsep}{0pt}\setlength{\parskip}{0pt}}\usepackage{longtable,booktabs,array}
\usepackage{calc} % for calculating minipage widths
\usepackage{caption}
% Make caption package work with longtable
\makeatletter
\def\fnum@table{\tablename~\thetable}
\makeatother
\usepackage{graphicx}
\makeatletter
\newsavebox\pandoc@box
\newcommand*\pandocbounded[1]{% scales image to fit in text height/width
  \sbox\pandoc@box{#1}%
  \Gscale@div\@tempa{\textheight}{\dimexpr\ht\pandoc@box+\dp\pandoc@box\relax}%
  \Gscale@div\@tempb{\linewidth}{\wd\pandoc@box}%
  \ifdim\@tempb\p@<\@tempa\p@\let\@tempa\@tempb\fi% select the smaller of both
  \ifdim\@tempa\p@<\p@\scalebox{\@tempa}{\usebox\pandoc@box}%
  \else\usebox{\pandoc@box}%
  \fi%
}
% Set default figure placement to htbp
\def\fps@figure{htbp}
\makeatother

\makeatletter
\@ifpackageloaded{caption}{}{\usepackage{caption}}
\AtBeginDocument{%
\ifdefined\contentsname
  \renewcommand*\contentsname{Table of contents}
\else
  \newcommand\contentsname{Table of contents}
\fi
\ifdefined\listfigurename
  \renewcommand*\listfigurename{List of Figures}
\else
  \newcommand\listfigurename{List of Figures}
\fi
\ifdefined\listtablename
  \renewcommand*\listtablename{List of Tables}
\else
  \newcommand\listtablename{List of Tables}
\fi
\ifdefined\figurename
  \renewcommand*\figurename{Figure}
\else
  \newcommand\figurename{Figure}
\fi
\ifdefined\tablename
  \renewcommand*\tablename{Table}
\else
  \newcommand\tablename{Table}
\fi
}
\@ifpackageloaded{float}{}{\usepackage{float}}
\floatstyle{ruled}
\@ifundefined{c@chapter}{\newfloat{codelisting}{h}{lop}}{\newfloat{codelisting}{h}{lop}[chapter]}
\floatname{codelisting}{Listing}
\newcommand*\listoflistings{\listof{codelisting}{List of Listings}}
\makeatother
\makeatletter
\makeatother
\makeatletter
\@ifpackageloaded{caption}{}{\usepackage{caption}}
\@ifpackageloaded{subcaption}{}{\usepackage{subcaption}}
\makeatother

\usepackage{bookmark}

\IfFileExists{xurl.sty}{\usepackage{xurl}}{} % add URL line breaks if available
\urlstyle{same} % disable monospaced font for URLs
\hypersetup{
  pdftitle={Project S: Onboarding},
  pdfauthor={Job Schepens; Luke Günther},
  hidelinks,
  pdfcreator={LaTeX via pandoc}}


\title{Project S: Onboarding}
\author{Job Schepens \and Luke Günther}
\date{2025-05-15}

\begin{document}
\frame{\titlepage}


\begin{frame}[fragile]{Project S: In a nutshell}
\phantomsection\label{project-s-in-a-nutshell}
\vspace{.25cm}

\begin{enumerate}
\tightlist
\item
  \textbf{Research Data Management}:\\
  collection, curation, publication
\item
  \textbf{Experimental Design \& Statistical Analysis}:\\
  planning, implementation, interpretation
\item
  \textbf{Technical Support}:\\
  tool infrastructure, programming (R, Python)
\end{enumerate}

\vspace{.5cm}

\begin{columns}[T,onlytextwidth]
\begin{column}{0.48\linewidth}
\textbf{Our door is always open!}

2nd floor @ House of Prominence

\vspace{.5cm}

\textbf{Additional resources:}

\href{https://uni-koeln.sciebo.de/s/fvdR9aesnq7O5xu}{\texttt{S\_FAQ}
folder on Sciebo}
\end{column}

\hspace{1cm}

\begin{column}{0.48\linewidth}
\textbf{We offer}

\begin{itemize}
\tightlist
\item
  1:1 meetings
\item
  Group meetings\\
  (by project/interest)
\item
  Workshops
\end{itemize}
\end{column}
\end{columns}
\end{frame}

\begin{frame}[fragile]{Data Management}
\phantomsection\label{data-management}
\textbf{Sciebo} is our collaborative data cloud.\\
This is where all non-sensitive project data should be stored.

\begin{tikzpicture}[remember picture,overlay]
    \node[xshift=-1.5cm,yshift=-2.25cm] at (current page.north east) {\includegraphics[width=1cm]{sciebo-logo.png}};
\end{tikzpicture}

\vspace*{-1cm}

If you already have access to your University account, please make sure
to \textbf{activate} your Sciebo account as well:

\begin{enumerate}
\tightlist
\item
  Go to
  \href{https://hochschulcloud.nrw/en/index.html}{\textbf{https://hochschulcloud.nrw}}
\item
  Click on \texttt{Registration}
\item
  Select \texttt{Universität\ zu\ Köln}
\item
  Log in with your University credentials
\end{enumerate}

You can then access your files at
\href{https://uni-koeln.sciebo.de/login}{\textbf{https://uni-koeln.sciebo.de}}
\end{frame}

\begin{frame}{Upcoming Events}
\phantomsection\label{upcoming-events}
\begin{itemize}
\tightlist
\item
  \href{https://sfb1252.uni-koeln.de/en/early-career/events/workshops/research-data-and-methods}{\textbf{Research
  Data \& Methods}}

  \begin{itemize}
  \tightlist
  \item
    21 May \textbar{} \emph{Research Data Management} by Felix Rau\\
    (Data Center for the Humanities)
  \end{itemize}
\item
  \href{https://lehre.idh.uni-koeln.de/lehrveranstaltungen/sommersemester-2025/digital-humanities-cologne/}{\textbf{Digital
  Humanities Colloquium}}

  \begin{itemize}
  \tightlist
  \item
    22 May \textbar{} \emph{Adapting Language Models for the Analysis of
    Real World Textual Data} by Roman Klinger (Uni Bamberg)
  \end{itemize}
\item
  \href{https://ub.uni-koeln.de/en/courses-consultations/specials/reproducibilitea-in-the-humaniteas}{\textbf{ReproducibiliTea
  in the HumaniTeas}}

  \begin{itemize}
  \tightlist
  \item
    26 May \textbar{} \emph{Reproducibility when working with large
    language models: A hallucination?} by Nils Reiter (Project C11 \& S)
  \end{itemize}
\item
  \href{https://vielfalt.uni-koeln.de/en/news/diversity-week-you-make-the-difference}{\textbf{Diversity
  Week}}

  \begin{itemize}
  \tightlist
  \item
    3 June \textbar{} FLINTA* discussion group on learning to code\\
    by Elen Le Foll (Project B10)
  \end{itemize}
\end{itemize}
\end{frame}

\begin{frame}[standout]{}
\phantomsection\label{section}
Questions?
\end{frame}




\end{document}
